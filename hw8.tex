\documentclass[11pt]{article}
\usepackage{amsmath,textcomp,amssymb,geometry,graphicx,enumerate}

\def\Name{Steffan Voges}  % Your name
\def\SID{23434518}  % Your student ID number
\def\Login{cs170-cz} % Your login (your class account, cs170-xy)
\def\Homework{8}%Number of Homework
\def\Session{Fall 2014}


\title{CS170--Fall 2014 --- Solutions to Homework \Homework}
\author{\Name, SID \SID, \texttt{\Login}}
\markboth{CS170--\Session\  Homework \Homework\ \Name}{CS170--\Session\ Homework \Homework\ \Name, \texttt{\Login}}
\pagestyle{myheadings}

\newenvironment{qparts}{\begin{enumerate}[{(}a{)}]}{\end{enumerate}}
\def\endproofmark{$\Box$}
\newenvironment{proof}{\par{\bf Proof}:}{\endproofmark\smallskip}

\textheight=9in
\textwidth=6.5in
\topmargin=-.75in
\oddsidemargin=0.25in
\evensidemargin=0.25in


\begin{document}
\maketitle

\noindent
Collaborators: Ryan Flynn


\section*{1. Subsequence}
\noindent
\textbf{Main idea.}
The main idea is to iterate through B, while keeping track of whether or not each letter in A has been hit yet.  


\noindent
\textbf{Pseudocode.}\\
def algorithm(A[1...n], B[1...m]): \\
\indent for i = 1, 2, ..., m \\
\indent\indent if A[0] == B[i]: \\
\indent\indent\indent A = A[1:] \\
\indent if A is empty: return True \\
\indent else: return False

\noindent
\textbf{Proof of correctness.}
YOUR ANSWER GOES HERE

\noindent
\textbf{Running time.}
$O(m + n)$


\noindent
\textbf{Justification of running time.}
You complete m iterations through B, complete n searches through A.  Since each search through A is only complete once during an iteration of B, our running time is $O(m + n)$. 

\newpage
\section*{2. Another scheduling problem}
\noindent
\textbf{Main idea.}
YOUR ANSWER GOES HERE


\noindent
\textbf{Pseudocode.}
YOUR ANSWER GOES HERE

\noindent
\textbf{Proof of correctness.}
YOUR ANSWER GOES HERE


\noindent
\textbf{Running time.}
YOUR ANSWER GOES HERE


\noindent
\textbf{Justification of running time.}
YOUR ANSWER GOES HERE



\newpage
\section*{3. Park Tours}
\noindent
\textbf{Main idea.}
YOUR ANSWER GOES HERE


\noindent
\textbf{Pseudocode.}
YOUR ANSWER GOES HERE

\noindent
\textbf{Proof of correctness.}
YOUR ANSWER GOES HERE


\noindent
\textbf{Running time.}
YOUR ANSWER GOES HERE


\noindent
\textbf{Justification of running time.}
YOUR ANSWER GOES HERE


\newpage
\section*{4. Optimal binary search trees}
\noindent
\textbf{Main idea.}
YOUR ANSWER GOES HERE


\noindent
\textbf{Pseudocode.}
YOUR ANSWER GOES HERE

\noindent
\textbf{Proof of correctness.}
YOUR ANSWER GOES HERE


\noindent
\textbf{Running time.}
YOUR ANSWER GOES HERE


\noindent
\textbf{Justification of running time.}
YOUR ANSWER GOES HERE


\newpage
\section*{5. Beat inference}
\noindent
\textbf{Main idea.}
YOUR ANSWER GOES HERE


\noindent
\textbf{Pseudocode.}
YOUR ANSWER GOES HERE

\noindent
\textbf{Proof of correctness.}
YOUR ANSWER GOES HERE


\noindent
\textbf{Running time.}
YOUR ANSWER GOES HERE


\noindent
\textbf{Justification of running time.}
YOUR ANSWER GOES HERE


\newpage
\section*{6. Optional Bonus Problem: Image re-sizing}
\noindent
\textbf{Main idea.}
YOUR ANSWER GOES HERE


\noindent
\textbf{Pseudocode.}
YOUR ANSWER GOES HERE

\noindent
\textbf{Proof of correctness.}
YOUR ANSWER GOES HERE


\noindent
\textbf{Running time.}
YOUR ANSWER GOES HERE


\noindent
\textbf{Justification of running time.}
YOUR ANSWER GOES HERE



\end{document}
